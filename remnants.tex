

The exponential term becomes subdominant to
$\pi\sqrt{\psi}/8\Omega$, and overlap occurs if
\begin{equation}
  \label{eq:touchlow}
   \left[  {E_{r0}\over\psi}
%    {\sqrt{\psi}\over \Omega}
%    {16\over\pi}
     {4\over b^2\pi}
    {v_\perp\over\sqrt\psi}\right]^{1/2}
  \gtrsim  [n(n+2)]^{-3/4},
\end{equation}
which for $n=2$ is
\begin{equation}
  \label{eq:lowbn2}
 {1\over L_\perp}={E_{r0}\over\psi} \gtrsim 
%    {\pi\over 16\times8^{3/2}}  
    {\pi b\over 4\times8^{3/2}}  
    {\Omega\over v_\perp}
    ={b\over 29\rho},
\end{equation}
where $\rho$ is the gyro radius. Thus, at small $b$, no
positive-$W$ resonant orbits are permanently trapped unless the
transverse scale length is very large:
$L_\perp \gtrsim 29 \rho_\perp/b= 29\rho_\perp^2 (\psi/v_\perp)$.
However, there are no deeply-trapped low-lying resonances
(island-centers) in this case, because $\wr=4b^2/n^2\ll 1$; so it is
possible for deep orbits to be permanently trapped, regardless of
island overlap. Our analysis above gives the shallowest detrapped
orbit energy as 
\begin{equation}
  \label{eq:lowb}
  \sqrt{\wp}=%\left(
  b+
   \left[  {E_{r0}\over\psi}
     {4\over \pi}
    {v_\perp\over\sqrt\psi}\right]^{1/2},
%\right)
\end{equation}
in which the second term becomes predominant at low $b$.


For example, the $\wr=0.2$ island of Fig.\ 
\ref{fig:lowbislands} corresponds to $b=\sqrt{0.2}=0.45$. Then 29/b=65,
eq. \ref{eq:lowbn2} becomes $L_\perp\lesssim 65\rho_\perp$. And when
$w=\psi$, so $\rho_\perp=\sqrt{w+w_\perp}=1.1$ Yet the orbits
with $W_\parallel/\psi<-0.2$ are probably permanently trapped.
\begin{figure}[htp]
  \centering
  \includegraphics[width=0.5\hsize]{lowbislands}
  \caption{Islands for moderately low magnetic field
    $b=\Omega/\sqrt{\psi}=0.45$ do not extend far into the potential
    well, even if their harmonics overlap. ($\psi=1$, $w=1$, $E_{r0}=0.01$.)}
  \label{fig:lowbislands}
\end{figure}


This expression allows us to calculate a condition for adjacent
islands to ``touch'' (their separatrices to marginally
overlap). Touching occurs between the $n$ and $n+2$ harmonics when:
$\sqrt{w_{\parallel R n}}-\delta_n=\sqrt{w_{\parallel R
    n+2}}+\delta_{n+2}$, which becomes 
\begin{equation}
  \label{eq:touch}
  2b\left({1\over n}-{1\over n+2}\right) 
  =\delta_n+ \delta_{n+2}
  \simeq\left({1\over\sqrt{n}}+{1\over\sqrt{n+2}}\right)\delta.
\end{equation}
Approximating the arithmetic mean of $\sqrt{n}$ and $\sqrt{n+2}$ by
their geometric mean, gives $({1\over n}-{1\over n+2})/({1\over\sqrt
  n}+{1\over\sqrt{n+2}})\simeq [n(n+2)]^{-3/4} $ and 
the criterion for island overlap becomes,
\begin{equation}
  \label{eq:touchmean}
  {\delta\over 2b}=
  \left[  {E_{r0}\over\psi}
   {v_\perp\over\sqrt\psi}{1\over 
     2b^2(b{\rm e}^b+\pi/8)
   }\right]^{1/2}
  \gtrsim  [n(n+2)]^{-3/4}
\end{equation}
For the lowest ($n$) island satisfying this condition yet still
present in the relevant energy range
($-W_{\parallel R}/\sqrt\psi=\wr\le 1$, i.e.\ $n\ge2b$), its lower
$W_\parallel$ extreme (denoted $W_{\parallel t}$) has
\begin{equation}\label{eq:wpt}
\sqrt{-W_{\parallel t}/\psi}=\sqrt{w_{\parallel t}} = (\sqrt{w_{\parallel R n}}+\delta_n)={2b\over
  n}(1+{\delta\over 2b}\sqrt{n}).
\end{equation}


%%%%%%%%%%%%%%%%%%%%%%%%%%%%%%%%%%%%%%%%%%%%%%%%%%%%%%%%%%%%%%%%%%%%%%%%%%%
\hrule
\hrule
\section{Old Calculations}
\subsection{Old Shallow calculation}

Substituted into eq.\ (\ref{eq:orbitave}), this gives the required
form for $dW_\parallel/dt$.

Substituting for $\omega_n=n\omega_b=n\sqrt{|W_\parallel|/2}$, giving
\begin{equation}
  \label{eq:xideriv}
  {d\xi\over dt} =
  \omega_n-\Omega={n\over\sqrt{2}}(\sqrt{-W_\parallel} -
  \sqrt{W_{\parallel R}}),
\end{equation}
 and
$v_\perp=\sqrt{2(W-W_\parallel)}$,
we get the $W_\parallel$ trajectory
\begin{equation}\label{eq:dwdt}
  {dW_\parallel\over dt} = {n\over\sqrt{2}}(\sqrt{-W_\parallel} -
  \sqrt{-W_{\parallel R}}) {dW_\parallel\over d\xi} 
  = - {A \over \pi}\sqrt{W-W_\parallel}\sqrt{-W_\parallel}\cos\xi.
\end{equation}

Then
\begin{equation}
  \label{eq:shalnorm}
  {n\pi\over \sqrt{2}}
  {\sqrt{\wp}-\sqrt{\wr }\over
    \sqrt{\wp}\sqrt{w+\wp}} {d\wp\over d\xi} 
  ={A\over\sqrt{\psi}}\cos\xi \simeq {8E_{r0}\over\sqrt{2}\psi }\cos\xi ,
\end{equation}
where $\wr $ is the value of $\wp$ at exact
resonance $n\omega_b=\Omega$,
$\sqrt{\wr }=\sqrt{2}(\Omega/n\sqrt{\psi})$.

If $A$ is taken as a constant, this equation can
be integrated giving 
\begin{equation}
  \label{eq:constA}
  2\sqrt{w+\wp}
  -2\sqrt{\wr }\ln(\sqrt{\wp}+\sqrt{w+\wp})
  -{A\over\sqrt{\psi}}{\sqrt{2}\over n\pi}\sin\xi = const.,
\end{equation}
This is a conveniently universal form. In addition to the variables
$\wp$ and $\xi$, $A/\sqrt{\psi}$ represents the strength of
the perturbation, and $\wr $ represents the magnetic field
strength (divided by $n$). The total energy is assumed positive
(otherwise we already know the orbit is permanently trapped), and in
fact we care most, for shallowly trapped particles, about situations
in which $w\gg -\wp$.  In that case, Taylor expansion of the
$\wp$ dependence is sufficient and gives the analytically
simpler expression
\begin{equation}
  \label{eq:taylorw}
  (\sqrt{\wp}-\sqrt{\wr })^2 + 
{A\sqrt{w}\over\sqrt{\psi}}{\sqrt{2}\over n\pi}\sin\xi \simeq const.
\end{equation}
And we can immediately conclude that the separatrix minima and maxima are
\begin{equation}
\sqrt{\wp}-\sqrt{\wr }=\pm
\sqrt{2\sqrt{2}A\sqrt{w}\over \pi\sqrt{\psi}n}
\simeq\pm \sqrt{{16 \over  n\pi}{E_{r0}\over \psi}}.
\end{equation}

Islands corresponding to adjacent (even) harmonics $n$ and $n+2$
touch, giving the threshold of stochasticity, when 
\begin{equation}
  \label{eq:touching}
  \sqrt{w_{\parallel R_n}}-\sqrt{w_{\parallel R_{n+2}}}=\sqrt{2\sqrt{2}A\sqrt{w}\over \pi\sqrt{\psi}}\left({1\over\sqrt{n}}+{1\over\sqrt{n+2}}\right)
\end{equation}
Then, since $w_{\parallel R_n}=(2/\psi)\Omega^2/n^2$, and
$A=8E_{r0}/\sqrt{2\psi}$, marginal overlap
occurs when
\begin{equation}
  \label{eq:Kmarginal}
  K\equiv
  \left[{\sqrt{W\over\psi}{\psi\over\Omega^2}{8\sqrt{2}\over \pi}{E_{r0}\over\psi}}\right]^{1/2}
  = \left({1\over n}-{1\over n+2}\right)\Big/
  \left({1\over\sqrt{n}}+{1\over\sqrt{n+2}}\right)
  \simeq \left[n(n+2)\right]^{-3/4},
\end{equation}
where the final (rather accurate) convenient approximation replaces the
arithmetic mean of $\sqrt{n}$ and $\sqrt{n+2}$ with their geometric
mean. When the left hand side ($K$)
is greater than the right hand side, harmonic $n$ will have (at best)
stochastic $\wp$ trajectories, and so will all higher
harmonics, indicating they are not permanently trapped.
Permanent trapping requires a small enough value of
$K$. We can also invert this expression to find a
critical integer value $n=n_c$ below which orbits are permanently
trapped as
\begin{equation}
  \label{eq:ncrit}
  n_c=-1+\sqrt{1+K^{-4/3}}
\end{equation}
(rounded down). And the resonant parallel energy corresponding to
$n_c$ is $-\wp =2\Omega^2/n_c^2$. The lowest harmonic is
$n_c=2$, for which $K=8^{-3/4}$. For higher values of $K$ there are no
permanently trapped resonant orbits in the shallow orbit
approximation. 

For the resonance to be shallow, i.e.\ to lie at small
$-\wp/\psi$, one requires that
$-W_{\parallel R}=2\Omega^2/n\ll \psi$ so $\Omega^2/n\psi\ll
1$.
Consequently this calculation is appropriate for low harmonics $n$
only for weak magnetic field, meaning cyclotron frequency much less
than the bounce frequency of deeply trapped orbits $\sqrt{\psi}/2$.
If, instead, $\Omega^2/\psi$ is of order unity, then only high
harmonics will be appropriately treated using the shallow trapping
approximations. In that case, to avoid stochastic island overlap
requires $K$ to be smaller, and the requirement may be rewritten,
taking $n=\Omega/\sqrt{-W_{\parallel R}/2}$, as 
\begin{equation}
  \label{eq:highharm}
  {E_{r0}\over\psi}  \lesssim
  {(-W_{\parallel R}/2)^{3/2}\over\Omega^3}\sqrt{\psi\over W} 
  {\Omega^2\over\psi} {\pi\over 8\sqrt2} =
  {\pi\over 32}
  {(-W_{\parallel R})^{3/2}\over\Omega\sqrt{\psi W}} 
  \sim {\pi\over 32} {(-W_{\parallel R}/\psi)^{3/2}\over \sqrt{W/\psi}}
\end{equation}

Notice that $E_{r0}/\psi=-{d\psi\over dr}/\psi=L_\perp$ is the inverse
scale length of the perpendicular potential variation. So this can be
considered a demonstration that shallow orbits of energy greater than
$\sim-\psi [(32/\pi)\sqrt{W/\psi}/L_\perp]^{2/3}$ are
stochastic. Supposing that the total energy $W$ has the typical order
of magnitude of unity (thermal energy), and noting
$(32/\pi)^{2/3}=4.7$, we conclude that permanently trapped orbits when
$\Omega^2\simeq\psi$ lie only below a parallel energy of
$\sim -5 \psi^{1/3}/L_\perp^{2/3}$.  Higher parallel energy orbits
(closer to $\wp=0$) cannot have their phase-space density $f$
significantly depressed below the untrapped value at
$\wp \to 0+$, and so cannot contribute to the positive charge
required to sustain the hole potential.

\subsection{Old Deep Calculation}

A plausible interpolation between the two $\wp$ limits is to 
compare the above with
\begin{equation}
  \label{eq:E2int}
  E_n= E_{r0}{16\over
    \pi\sqrt{2\psi}}\sqrt{-\wp/2}
\end{equation}
and then to write
\begin{equation}
  \label{eq:E2int2}
  E_2\simeq E_{r0}
  \left[
    {2\over 1+\wp/\psi}+ 
    {\pi\over 8}\sqrt{\psi\over -\wp}\,
  \right]^{-1}.
\end{equation}
OR maybe we incorporate the appropriate limit of $\omega_2$ into the
two terms here.

The resulting $\wp$ trajectory equation is
\begin{equation}
  \label{eq:deeptraj}
  {d\wp \over dt}=  (\omega_2-\Omega){d\wp \over d\xi}
  =-E_2 v_\perp \cos\xi 
  =-{1\over\sqrt{2}}E_{r0}(1+\wp/\psi)\sqrt{W-\wp}\cos\xi . 
\end{equation}
It shows $E _2\to 0$ at
the bottom of the well, confirming that $\wp$ trajectories crowd
together near $\wp=-\psi$ (and do not cross this level, no
matter how wide the trajectory island is). 


Into this equation we must substitute the $\omega_2$ dependence, which
can reasonably be approximated in the deep trapping regime as 
\begin{equation}
  \label{eq:omegadeep}
  \omega_2=\sqrt{-\wp},
\end{equation}
giving $\sqrt{\wr }=\Omega/\sqrt\psi$, and

\begin{equation}
  \label{eq:deepWtraj}
  {\sqrt{-\wp}-\Omega \over(1+\wp/\psi)\sqrt{W-\wp}}
  {d\wp \over d\xi}
  =-{1\over2}E_{r0}\cos\xi.
\end{equation}


\begin{equation}
  \label{eq:deepWtraj}
2  {\sqrt{\wp}-\sqrt{\wr } \over
    (1-\wp)\sqrt{w+\wp}}
  {d\wp \over d\xi}
  ={E_{r0}\over\psi}\cos\xi.
\end{equation}
This can be integrated as
\begin{equation}
  \label{eq:deepinteg}
\begin{split}
  {4\over\sqrt{w+1}}\Bigg[\tanh^{-1}\sqrt{\wp(w+1)\over
    w+\wp} &-\sqrt{w+1}\ln(\sqrt{w+\wp}+\sqrt{\wp})\\
  &-\sqrt{\wr }\tanh^{-1}\sqrt{w+\wp\over w+1}
  \Bigg] = {E_{r0}\over \psi}\sin\xi.
\end{split}
\end{equation}

\subsection{Interpolated Expression}


The
form we will adopt for analytical convenience is
\begin{equation}
  \label{eq:E2int}
  E_2= E_r\left
[{2\over 1-\wp} + {\pi\over8}{1\over\sqrt{\wp}}\right]^{-1},
\end{equation}
which agrees with eqs.\ \ref{eq:fouriermodes} and \ref{eq:E2} in their
respective limits.


Let us rewrite the shallow orbit expression, eq.\ (\ref{eq:dwdt}), in
the same form as (\ref{eq:deepWtraj}), using $A=8E_{r0}/\sqrt{2\psi}$,
and $\sqrt{\wr }=\sqrt{2}\omega_b/\sqrt\psi=(\sqrt{2}/n)\Omega/\sqrt{\psi}$ to
obtain
\begin{equation}
  \label{eq:shallowWtraj}
   {n\pi\over 8}{\sqrt{\wp}-(\sqrt{2}/n)\Omega/\sqrt{\psi} 
     \over\sqrt{\wp}\sqrt{w+\wp}}
  {d\wp \over d\xi}
  ={E_{r0}\over\psi}\cos\xi.
\end{equation}
Now observe that the equations for deep and shallow trapping each
possess a coefficient of $d\wp/d\xi$ that becomes large in
their respective $\wp$ limits. It is then clear that one can
interpolate a smooth solution that will be correct in the limits by simply
putting the \emph{sum} of the left hand sides equal to the (unsummed)
right hand side. There is no guarantee that the interpolation will be
very accurate for intermediate values of $-\wp$; but in any
case we have no guarantee that the deep trapping solution is precise
for every electron hole profile; so there are other significant
quantitative uncertainties in the treatment. This approximation permits a
closed-form solution of the $\wp$ trajectory because (writing
$x=-\wp/\psi$, $a=W/\psi$)
\begin{align}
  \label{eq:integrals}
  \int {dx\over(1-x)\sqrt{a+x}}
  &={2\over\sqrt{a+1}}\tanh^{-1}\sqrt{a+x\over a+1};\\
  \int {\sqrt{x}dx\over(1-x)\sqrt{a+x}}
  &= {2\over\sqrt{a+1}}\tanh^{-1}\sqrt{x(a+1)\over (a+x)}
- 2\ln(\sqrt{a+x}+\sqrt{x});\\
  \int {dx\over\sqrt{a+x}} 
  &= 2\sqrt{a+x};\\
  \int {dx\over\sqrt{x}\sqrt{a+x}}
  &= 2\ln(\sqrt{a+x}+\sqrt{x}).
\end{align}
Hence we have the trajectories 
\begin{equation}
  \label{eq:interpint}
\begin{split}
  &\Bigg[{2\over\sqrt{W/\psi+1}}
      \tanh^{-1}\sqrt{(-\wp/W)(W/\psi+1)\over W/\psi-\wp/W}
      - 2\ln\left(\sqrt{{W\over\psi}-{\wp\over W}}
      +\sqrt{-\wp\over W}\right)\\
&\ -{\Omega\over\sqrt{\psi}}{2\over\sqrt{W/\psi+1}}\tanh^{-1}\sqrt{W/\psi-\wp/W\over
   W/\psi+1}\\
&\ +{\pi\over 8}\Bigg\{2\sqrt{{W\over\psi}-{\wp\over W}}
   -{\sqrt{2}\Omega\over\sqrt{\psi}}
   2\ln\left(\sqrt{{W\over\psi}-{\wp\over W}}
   +\sqrt{-\wp\over W}\right)\Bigg\}
\Bigg]_{W_{\parallel R}}^{W_{\parallel}}
   -{{1\over2}}E_{r0}\sin\xi  = {\rm C}.
\end{split}
\end{equation}
If, as indicated, we take the definite integral (in the square
brackets) to be from the resonant parallel energy $W_{\parallel r}$,
then the value of the integration constant at the trajectory
separatrix is $C=-{1\over2}E_{r0}$, making the definite integral zero
at the x-point $\xi=\pi/2$. The width of the separatrix at its maximum
($\xi=-\pi/2$) is the solution when the definite integral is equal to
$-E_{r0}$.

The easiest way to plot the trajectories is as contours of the value
of the left hand side.



\section{Old Function details}

The integrated expressions for $F_n$ are as follows
\begin{align}
  \label{eq:integrals}
  F_n&={n^2\over 4\sqrt{2}}g_n+{n\pi\over16\sqrt2} g_0,\\
\noalign{with}
  g_0&=\int {(\sqrt{\wp}-\sqrt{\wr })d\wp
       \over\sqrt{w+\wp}\sqrt{\wp}} 
       = 2\sqrt{w+\wp}- 
       2\sqrt{w_{\parallel_R}}h_1;\\
  g_n&=\int  {(\sqrt{\wp}-\sqrt{\wr })d\wp
       \over\sqrt{w+\wp}(1-\sqrt{\wp})^{n/2}}
       = g_{n0}+a_{n1}h_1+a_{n2}h_2,\\
\noalign{ where the $h$ functions are}
  h_1&(w,\wp)=\ln(\sqrt{w+\wp}+\sqrt{\wp})\nonumber\\
  h_2&(w,\wp)=\ln(1-\sqrt{\wp}) 
       -\ln(\sqrt{w+1}\sqrt{w+\wp}+w+\sqrt{\wp})\\
\noalign{and}
  g_{20}&=-2\sqrt{w+\wp},\quad a_{21}=2(\sqrt{\wr }-1),
          \quad a_{22}=2(\sqrt{\wr }-1)/\sqrt{w+\wp};\nonumber\\
  g_{40}&=-{2(\sqrt{\wr }-1)\sqrt{w+\wp}\over 
          (w+1)(1-\sqrt{\wp})},\quad a_{41}=2,\quad 
          a_{42}=2{w(\sqrt{\wr }-2)-1\over (w+1)^{3/2} };\nonumber\\
  g_{60}&={\{w[\sqrt{\wr }(1-2\sqrt{\wp})
          +4\sqrt{\wp}-3]
          +\sqrt{\wr }(\sqrt{\wp}-2)+\sqrt{\wp}
          \}\sqrt{w+\wp}\over
          (w+1)^2(1-\sqrt{\wp})^2},\nonumber\\
  &\qquad a_{61}=0,\quad a_{62}={w(2w+3\sqrt{\wr }-1) \over
          (w+1)^{5/2}};\\
    g_{80}&=\Big\{-(1-\sqrt{\wp})^2
            [-6w^2+w(10-13\sqrt{\wr })
            +2\sqrt{\wr }+1]\nonumber\\
     &\qquad +(w+1)(1-\sqrt{\wp})[3w(\sqrt{w_{\parallel
       R}}-2)-2\sqrt{\wr }-1 ] \nonumber\\
     &\qquad  +2(w+1)^2(1-\sqrt{\wr })\nonumber\\
 &\qquad
   \Big\}\sqrt{w+\wp}\Big/[3(w+1)^3(1-\sqrt{\wp})^3],\nonumber\\
   &\qquad a_{81}=0,\quad 
     a_{82}= {w[w(\sqrt{\wr }-4)-4\sqrt{\wr }+1]
     \over (w+1)^{7/2}}.\nonumber
\end{align}
Harmonic numbers $n$ higher than 8 have increasing algebraic
complexity of the denominators ($\{\dots\}$) of $g_{n0}$ and $a_{n2}$,
but the $a_{n1}$ are zero.


%%% Local Variables:
%%% mode: latex
%%% TeX-master: t
%%% End:
